\documentclass{article}
\usepackage{fancyhdr}
\usepackage{extramarks}
\usepackage{amsmath}
\usepackage{amsthm}
\usepackage{amsfonts}
\usepackage{array}
\usepackage{tikz}
\usepackage[plain]{algorithm}
\usepackage{algpseudocode}
\usepackage{enumitem}
\usepackage{pgfplots}
\usepackage{steinmetz}
\usepackage{siunitx}
\usepackage{subcaption}
\pgfplotsset{}
\usepackage{tabularx}
\usepackage{marginnote}

\topmargin=-0.45in
\evensidemargin=0cm
\oddsidemargin=0in
\textwidth=6.5in
\textheight=9.0in
\headsep=0.25in


\linespread{1.1}

\pagestyle{fancy}
\renewcommand\headrulewidth{0.4pt}
\renewcommand\footrulewidth{0.4pt}

\setlength\parindent{0pt}
\newcolumntype{C}[1]{>{\centering\arraybackslash}p{#1}}
\newcolumntype{R}[1]{>{\raggedleft\arraybackslash}p{#1}}
\setlength{\tabcolsep}{0mm}

\lhead{Group P}
\chead{Computer Organization Project}
\rhead{ISA Introduction}

\title{
	\vspace{2in}
	\textmd{Computer Organization \\ Floating Point Coprocessor}\\
	\vspace{1cm}
	\textmd{\textbf{Instruction Set Architecture}}
	\vspace{5cm}
}

\author{Guillermo Brice\~{n}o \\ Rene Nicklich \\ Austin Lantz}



\begin{document}

\maketitle

\pagebreak

\section{Introduction}
This document describes the instruction set architecture for a 32-bit floating-point coprocessor. Each instruction is described with its assembler syntax in section 2.\\

\subsection{Instruction Encoding}
The twenty-three instructions can be grouped by their instruction encoding formats. The keywords \texttt{rd}, \texttt{rs1}, and \texttt{rs2} stand for the 4-bit register addresses from $\texttt{0000}_2$ to $\texttt{1111}_2$, corresponding to a destination register, the first source register, and second source register respectively. In single-source or immediate mode addressing, the keyword \texttt{rs} is used instead. Bit 31 is used to specify register or immediate addressing mode for double-operand ALU operations.

\normalsize
\bigskip

\begin{minipage}[t]{0.3\textwidth}
	\begin{center}
		\medskip
		\textbf{Double-Operand\\ ALU}
	\end{center}
\end{minipage}
\begin{minipage}[t]{110mm}
	\begin{tabular}{p{12mm}p{2.5mm}p{13mm}p{10mm}p{10mm}p{10mm}p{22.5mm}R{22.5mm}}
		& \footnotesize{} & \footnotesize{30} & \footnotesize{25} & \footnotesize{21} & \footnotesize{17} & \footnotesize{13} & \footnotesize{0} \\ 
	\end{tabular} 

	\begin{tabular}{p{12mm}|C{2.5mm}|C{12.5mm}|C{10mm}|C{10mm}|C{10mm}|C{45mm}|}
		\cline{2-7}
		r-type: & 0 & opcode & rd & rs1 & rs2 & 0\\ 
		\cline{2-7}
	\end{tabular} 
	
	\medskip

	\begin{tabular}{p{12mm}|C{2.5mm}|C{12.5mm}|C{10mm}|C{10mm}|C{55.15mm}|}
		\cline{2-6}
		i-type: & 1 & opcode & rd & rs & 18-bit immediate \\
		\cline{2-6}
	\end{tabular}
\end{minipage}

\bigskip\bigskip

\begin{minipage}[t]{0.3\textwidth}
	\begin{center}
		\textbf{Single-Operand\\ ALU}
	\end{center}
\end{minipage}
\begin{minipage}[t]{110mm}	
	\begin{tabular}{p{12mm}|C{2.5mm}|C{12.5mm}|C{10mm}|C{65.3mm}|}
		\cline{2-5}
		 EXP: & 0 & opcode & rd & 22-bit immediate \\
		\cline{2-5}
	\end{tabular}

	\medskip

	\begin{tabular}{p{12mm}|C{2.5mm}|C{12.5mm}|C{10mm}|C{65.3mm}|}
		\cline{2-5}
		 & 0 & opcode & rd & 0 \\
		\cline{2-5}
	\end{tabular}
\end{minipage}

\bigskip\bigskip\bigskip

\begin{minipage}[t]{0.3\textwidth}
	\begin{center}
		\textbf{Load, Store, \\ Move}
	\end{center}
\end{minipage}
\begin{minipage}[t]{110mm}

	\begin{tabular}{p{12mm}|C{2.5mm}|C{12.5mm}|C{10mm}|C{10mm}|C{55.15mm}|}
		\cline{2-6}
		 & 0 & opcode & rd & rs & 0 \\
		\cline{2-6}
	\end{tabular}

	\end{minipage}

\bigskip\bigskip\bigskip

\begin{minipage}[t]{0.3\textwidth}
	\begin{center}
		\textbf{Branching}
	\end{center}
\end{minipage}
\begin{minipage}[t]{110mm}
	\begin{tabular}{p{12mm}|C{2.5mm}|C{12.5mm}|C{10mm}|C{65.3mm}|}
		\cline{2-5}
		\quad B: & 0 & opcode & rd & 22-bit immediate \\
		\cline{2-5}
	\end{tabular}

	\medskip

	\begin{tabular}{p{12mm}|C{2.5mm}|C{12.5mm}|C{10mm}|C{65.3mm}|}
		\cline{2-5}
		 & 0 & opcode & rd & 0 \\
		\cline{2-5}
	\end{tabular}

\end{minipage}

\bigskip\bigskip\bigskip

\begin{minipage}[t]{0.3\textwidth}
	\begin{center}
		\textbf{No-Operation and \\ Halt}
	\end{center}
\end{minipage}
\begin{minipage}[t]{110mm}
	\begin{tabular}{p{12mm}|C{2.5mm}|C{12.5mm}|C{75.3mm}|}
		\cline{2-4}
		 & 0 & opcode &  0 \\
		\cline{2-4}
	\end{tabular}
\end{minipage}

\pagebreak

\rhead{ISA Specification}
\section{Instruction Set Specification}

\bigskip\bigskip

%NOP
\flushleft
\LARGE\textbf{NOP} \large \hfill No operation

\kern-3pt
\noindent\rule{16.5cm}{0.4pt}
\normalsize

{\large
	\makebox[3.5cm][l]{Opcode:} \texttt{00000} \par
	\smallbreak
	\makebox[3.5cm][l]{Syntax:} \texttt{NOP} \par
	\smallbreak
	\makebox[3.5cm][l]{Purpose:} Perform no operations. \par
}

\bigskip\bigskip

%SET
\flushleft
\LARGE\textbf{SET} \large \hfill Set register to floating-point value

\kern-3pt
\noindent\rule{16.5cm}{0.4pt}
\normalsize

{\large
	\makebox[3.5cm][l]{Opcode:} \texttt{00001} \par
	\smallbreak
	\makebox[3.5cm][l]{Syntax r-type:} \texttt{SET rd, \#<32-bit FP value>} \par
	\smallbreak
	\makebox[3.5cm][l]{Purpose:} Assign a 32-bit floating point value to rd. \par
	\smallbreak
	\makebox[3.5cm][l]{Operation:} rd $\leftarrow$ FPvalue \par
}

\bigskip\bigskip

%LOAD
\flushleft
\LARGE\textbf{LOAD} \large \hfill Load value from memory

\kern-3pt
\noindent\rule{16.5cm}{0.4pt}
\normalsize

{\large
	\makebox[3.5cm][l]{Opcode:} \texttt{00010} \par
	\smallbreak
	\makebox[3.5cm][l]{Syntax r-type:} \texttt{LOAD rd, rs} \par
	\smallbreak
	\makebox[3.5cm][l]{Purpose:} Assign rd the value from the memory address in rs. \par
	\smallbreak
	\makebox[3.5cm][l]{Operation:} rd $\leftarrow$ M[rs] \par
}

\bigskip\bigskip

%STORE
\flushleft
\LARGE\textbf{STORE} \large \hfill Store value to memory

\kern-3pt
\noindent\rule{16.5cm}{0.4pt}
\normalsize

{\large
	\makebox[3.5cm][l]{Opcode:} \texttt{00011} \par
	\smallbreak
	\makebox[3.5cm][l]{Syntax r-type:} \texttt{STORE rd, rs} \par
	\smallbreak
	\makebox[3.5cm][l]{Purpose:} Assign memory location specified in rd to value in rs. \par
	\smallbreak
	\makebox[3.5cm][l]{Operation:} M[rd] $\leftarrow$ rs \par
}

\bigskip\bigskip

%MOVE
\flushleft
\LARGE\textbf{MOVE} \large \hfill Copy value from a register to another

\kern-3pt
\noindent\rule{16.5cm}{0.4pt}
\normalsize

{\large
	\makebox[3.5cm][l]{Opcode:} \texttt{00100} \par
	\smallbreak
	\makebox[3.5cm][l]{Syntax r-type:} \texttt{MOVE rd, rs} \par
	\smallbreak
	\makebox[3.5cm][l]{Purpose:} Assign rd the value in rs. \par
	\smallbreak
	\makebox[3.5cm][l]{Operation:} rd $\leftarrow$ rs \par
}

\pagebreak

%FADD
\flushleft
\LARGE\textbf{FADD} \large \hfill Add 32-bit floating-point value

\kern-3pt
\noindent\rule{16.5cm}{0.4pt}
\normalsize

{\large
	\makebox[3.5cm][l]{Opcode:} \texttt{00101} \par
	\smallbreak
	\makebox[3.5cm][l]{Syntax r-type:} \texttt{FADD rd, rs1, rs2} \par
	\smallbreak
	\makebox[3.5cm][l]{Syntax i-type:} \texttt{FADD rd, rs1, \#<32-bit ? immediate>} \par
	\smallbreak
	\makebox[3.5cm][l]{Purpose:} Performs addition on two 32-bit floating-point values from rs1 and \par
	\makebox[3.5cm][l]{  } rs2, or an immediate in place of rs2, and stores the result in rd. \par
	\smallbreak
	\makebox[3.5cm][l]{Operation:} rd $\leftarrow$ rs1 + rs2 \quad or \quad rd $\leftarrow$ rs1 + immediate\par
	\smallbreak
	\makebox[3.5cm][l]{Condition Codes:} \begin{tabular}{lll} N \quad & Z \quad & V \\ \hline x & x & x \\ \end{tabular}
}

\bigskip\bigskip

%FSUB
\flushleft
\LARGE\textbf{FSUB} \large \hfill Subtract 32-bit floating-point values

\kern-3pt
\noindent\rule{16.5cm}{0.4pt}
\normalsize

{\large
	\makebox[3.5cm][l]{Opcode:} \texttt{00110} \par
	\smallbreak
	\makebox[3.5cm][l]{Syntax r-type:} \texttt{FSUB rd, rs1, rs2} \par
	\smallbreak
	\makebox[3.5cm][l]{Syntax i-type:} \texttt{FSUB rd, rs1, \#<32-bit ? immediate>} \par
	\smallbreak
	\makebox[3.5cm][l]{Purpose:} Performs subtraction on two 32-bit floating-point values from rs1 and \par
	\makebox[3.5cm][l]{  } rs2, or an immediate in place of rs2, and stores the result in rd. \par
	\smallbreak
	\makebox[3.5cm][l]{Operation:} rd $\leftarrow$ rs1 - rs2 \quad or \quad rd $\leftarrow$ rs1 - immediate\par
	\smallbreak
	\makebox[3.5cm][l]{Condition Codes:} \begin{tabular}{lll} N \quad & Z \quad & V \\ \hline x & x & x \\ \end{tabular}
}

\bigskip\bigskip

%FNEG
\flushleft
\LARGE\textbf{FNEG} \large \hfill Negate a 32-bit floating-point value

\kern-3pt
\noindent\rule{16.5cm}{0.4pt}
\normalsize

{\large
	\makebox[3.5cm][l]{Opcode:} \texttt{00111} \par
	\smallbreak
	\makebox[3.5cm][l]{Syntax:} \texttt{FNEG rd, rs} \par
	\smallbreak
	\makebox[3.5cm][l]{Purpose:} Performs negation on a 32-bit floating-point value from rs and \par
	\makebox[3.5cm][l]{  } stores the result in rd. \par
	\smallbreak
	\makebox[3.5cm][l]{Operation:} rd $\leftarrow$ -rs \par 
	\smallbreak
	\makebox[3.5cm][l]{Condition Codes:} \begin{tabular}{lll} N \quad & Z \quad & V \\ \hline x & x & - \\ \end{tabular}
}

\pagebreak

%FMUL
\flushleft
\LARGE\textbf{FMUL} \large \hfill Multiply two 32-bit floating-point values

\kern-3pt
\noindent\rule{16.5cm}{0.4pt}
\normalsize

{\large
	\makebox[3.5cm][l]{Opcode:} \texttt{01000} \par
	\smallbreak
	\makebox[3.5cm][l]{Syntax r-type:} \texttt{FMUL rd, rs1, rs2} \par
	\smallbreak
	\makebox[3.5cm][l]{Syntax i-type:} \texttt{FMUL rd, rs1, \#<32-bit ? immediate>} \par
	\smallbreak
	\makebox[3.5cm][l]{Purpose:} Performs multiplication on two 32-bit floating-point values from rs1 \par
	\makebox[3.5cm][l]{  } and rs2, or an immediate in place of rs2, and stores the result in rd. \par
	\smallbreak
	\makebox[3.5cm][l]{Operation:} rd $\leftarrow$ rs1 $*$ rs2 \quad or \quad rd $\leftarrow$ rs1 $*$ immediate\par
	\smallbreak
	\makebox[3.5cm][l]{Condition Codes:} \begin{tabular}{lll} N \quad & Z \quad & V \\ \hline x & x & x \\ \end{tabular}
}

\bigskip\bigskip

%FDIV
\flushleft
\LARGE\textbf{FDIV} \large \hfill Divide two 32-bit floating-point values

\kern-3pt
\noindent\rule{16.5cm}{0.4pt}
\normalsize

{\large
	\makebox[3.5cm][l]{Opcode:} \texttt{01001} \par
	\smallbreak
	\makebox[3.5cm][l]{Syntax r-type:} \texttt{FDIV rd, rs1, rs2} \par
	\smallbreak
	\makebox[3.5cm][l]{Syntax i-type:} \texttt{FDIV rd, rs1, \#<32-bit ? immediate>} \par
	\smallbreak
	\makebox[3.5cm][l]{Purpose:} Performs division on two 32-bit floating-point values from rs1 \par
	\makebox[3.5cm][l]{  } and rs2, or an immediate in place of rs2, and stores the result in rd. \par
	\smallbreak
	\makebox[3.5cm][l]{Operation:} rd $\leftarrow$ rs1 $\div$ rs2 \quad or \quad rd $\leftarrow$ rs1 $\div$ immediate\par
	\smallbreak
	\makebox[3.5cm][l]{Condition Codes:} \begin{tabular}{lll} N \quad & Z \quad & V \\ \hline x & x & x \\ \end{tabular}
}

\bigskip\bigskip

%Floor
\flushleft
\LARGE\textbf{FLOOR} \large \hfill Compute the floor function

\kern-3pt
\noindent\rule{16.5cm}{0.4pt}
\normalsize

{\large
	\makebox[3.5cm][l]{Opcode:} \texttt{01010} \par
	\smallbreak
	\makebox[3.5cm][l]{Syntax r-type:} \texttt{FLOOR rd, rs} \par
	\smallbreak
	\makebox[3.5cm][l]{Purpose:} Rounds the value in rs to the nearest lowest integer and stores the \par
	\makebox[3.5cm][l]{  } result in rd. \par
	\smallbreak
	\makebox[3.5cm][l]{Operation:} rd $\leftarrow$ $\lfloor$rs$\rfloor$ \par
	\smallbreak
	\makebox[3.5cm][l]{Condition Codes:} \begin{tabular}{lll} N \quad & Z \quad & V \\ \hline x & x & x \\ \end{tabular}
}

\pagebreak

%Ceiling
\flushleft
\LARGE\textbf{CEIL} \large \hfill Compute the ceiling function

\kern-3pt
\noindent\rule{16.5cm}{0.4pt}
\normalsize

{\large
	\makebox[3.5cm][l]{Opcode:} \texttt{01011} \par
	\smallbreak
	\makebox[3.5cm][l]{Syntax:} \texttt{CEIL rd, rs} \par
	\smallbreak
	\makebox[3.5cm][l]{Purpose:} Rounds the value in rs to the nearest highest integer and stores the \par
	\makebox[3.5cm][l]{  } result in rd. \par
	\smallbreak
	\makebox[3.5cm][l]{Operation:} rd $\leftarrow$ $\lceil$rs$\rceil$ \par
	\smallbreak
	\makebox[3.5cm][l]{Condition Codes:} \begin{tabular}{lll} N \quad & Z \quad & V \\ \hline x & - & x \\ \end{tabular}
}

\bigskip\bigskip

%Round
\flushleft
\LARGE\textbf{ROUND} \large \hfill Round a value

\kern-3pt
\noindent\rule{16.5cm}{0.4pt}
\normalsize

{\large
	\makebox[3.5cm][l]{Opcode:} \texttt{01100} \par
	\smallbreak
	\makebox[3.5cm][l]{Syntax:} \texttt{ROUND rd, rs} \par
	\smallbreak
	\makebox[3.5cm][l]{Purpose:} Rounds the value in rs and stores the result in rd. \par
	\smallbreak
	\makebox[3.5cm][l]{Operation:} rd $\leftarrow$ round(rs) \par
	\smallbreak
	\makebox[3.5cm][l]{Condition Codes:} \begin{tabular}{lll} N \quad & Z \quad & V \\ \hline x & - & x \\ \end{tabular}
}

\bigskip\bigskip

%Absolute Value
\flushleft
\LARGE\textbf{FABS} \large \hfill Compute the absolute value

\kern-3pt
\noindent\rule{16.5cm}{0.4pt}
\normalsize

{\large
	\makebox[3.5cm][l]{Opcode:} \texttt{01101} \par
	\smallbreak
	\makebox[3.5cm][l]{Syntax:} \texttt{FABS rd, rs} \par
	\smallbreak
	\makebox[3.5cm][l]{Purpose:} Find the absolute value in rs and stores the result \par
	\makebox[3.5cm][l]{  } in rd. \par
	\smallbreak
	\makebox[3.5cm][l]{Operation:} rd $\leftarrow$ $|\textrm{rs}|$ \par
	\smallbreak
	\makebox[3.5cm][l]{Condition Codes:} \begin{tabular}{lll} N \quad & Z \quad & V \\ \hline x & - & x \\ \end{tabular}
}

\bigskip\bigskip

%Minimum
\flushleft
\LARGE\textbf{MIN} \large \hfill Find the smallest value

\kern-3pt
\noindent\rule{16.5cm}{0.4pt}
\normalsize

{\large
	\makebox[3.5cm][l]{Opcode:} \texttt{01110} \par
	\smallbreak
	\makebox[3.5cm][l]{Syntax:} \texttt{MIN rd, rs1, rs2} \par
	\smallbreak
	\makebox[3.5cm][l]{Purpose:} Finds the smallest value between rs1 and rs2 and stores \par
	\makebox[3.5cm][l]{  } the result in rd. \par
	\smallbreak
	\makebox[3.5cm][l]{Operation:} rd $\leftarrow$ min(rs1, rs2) \par
	\smallbreak
	\makebox[3.5cm][l]{Condition Codes:} \begin{tabular}{lll} N \quad & Z \quad & V \\ \hline x & - & x \\ \end{tabular}
}

\pagebreak

%Maximum
\flushleft
\LARGE\textbf{MAX} \large \hfill Find the largest

\kern-3pt
\noindent\rule{16.5cm}{0.4pt}
\normalsize

{\large
	\makebox[3.5cm][l]{Opcode:} \texttt{01111} \par
	\smallbreak
	\makebox[3.5cm][l]{Syntax:} \texttt{MAX rd, rs1, rs2} \par
	\smallbreak
	\makebox[3.5cm][l]{Purpose:} Finds the largest value between rs1 and rs2 and store the \par
	\makebox[3.5cm][l]{  } result in rd. \par
	\smallbreak
	\makebox[3.5cm][l]{Operation:} rd $\leftarrow$ max(rs1, rs2) \par
	\smallbreak
	\makebox[3.5cm][l]{Condition Codes:} \begin{tabular}{lll} N \quad & Z \quad & V \\ \hline x & - & x \\ \end{tabular}
}

\bigskip\bigskip

%Power
\flushleft
\LARGE\textbf{POW} \large \hfill Compute the power

\kern-3pt
\noindent\rule{16.5cm}{0.4pt}
\normalsize

{\large
	\makebox[3.5cm][l]{Opcode:} \texttt{10000} \par
	\smallbreak
	\makebox[3.5cm][l]{Syntax:} \texttt{POW rd, rs, \#<integer-value>} \par
	\smallbreak
	\makebox[3.5cm][l]{Purpose:} Finds rs1 raised to an integer value and stores the \par
	\makebox[3.5cm][l]{  } result in rd. \par
	\smallbreak
	\makebox[3.5cm][l]{Operation:} rd $\leftarrow$ $\textrm{rs}^{\textrm{integer-value}}$ \par
	\smallbreak
	\makebox[3.5cm][l]{Condition Codes:} \begin{tabular}{lll} N \quad & Z \quad & V \\ \hline x & - & x \\ \end{tabular}
}

\bigskip\bigskip

%Exponent
\flushleft
\LARGE\textbf{EXP} \large \hfill Compute the exponent

\kern-3pt
\noindent\rule{16.5cm}{0.4pt}
\normalsize

{\large
	\makebox[3.5cm][l]{Opcode:} \texttt{10001} \par
	\smallbreak
	\makebox[3.5cm][l]{Syntax:} \texttt{EXP rd, rs} \par
	\smallbreak
	\makebox[3.5cm][l]{Purpose:} Finds e raised to the value in rs and stores the \par
	\makebox[3.5cm][l]{  } result in rd. \par
	\smallbreak
	\makebox[3.5cm][l]{Operation:} rd $\leftarrow$ $e^{\textrm{rs}}$ \par
	\smallbreak
	\makebox[3.5cm][l]{Condition Codes:} \begin{tabular}{lll} N \quad & Z \quad & V \\ \hline x & - & x \\ \end{tabular}
}

\pagebreak

%Square Root
\flushleft
\LARGE\textbf{SQRT} \large \hfill Compute the square root of a value

\kern-3pt
\noindent\rule{16.5cm}{0.4pt}
\normalsize

{\large
	\makebox[3.5cm][l]{Opcode:} \texttt{10010} \par
	\smallbreak
	\makebox[3.5cm][l]{Syntax:} \texttt{SQRT rd, rs} \par
	\smallbreak
	\makebox[3.5cm][l]{Purpose:} Finds the square root of the value in rs and stores the\par
	\makebox[3.5cm][l]{  } result in rd. \par
	\smallbreak
	\makebox[3.5cm][l]{Operation:} rd $\leftarrow$ $\sqrt{\textrm{rs}}$ \par
	\smallbreak
	\makebox[3.5cm][l]{Condition Codes:} \begin{tabular}{lll} N \quad & Z \quad & V \\ \hline x & - & x \\ \end{tabular}
}

\bigskip\bigskip

%Unconditional Branch
\flushleft
\LARGE\textbf{B} \large \hfill Branch unconditionally

\kern-3pt
\noindent\rule{16.5cm}{0.4pt}
\normalsize

{\large
	\makebox[3.5cm][l]{Opcode:} \texttt{10011} \par
	\smallbreak
	\makebox[3.5cm][l]{Syntax:} \texttt{B rd} \par
	\smallbreak
	\makebox[3.5cm][l]{Purpose:} Set the program counter to the value in memory addressed  \par
	\makebox[3.5cm][l]{  } by rd. \par
	\smallbreak
	\makebox[3.5cm][l]{Operation:} PC $\leftarrow$ M[rd] \par
	\medbreak
	\makebox[3.5cm][l]{Condition Codes:} \begin{tabular}{lll} N \quad & Z \quad & V \\ \hline x & - & x \\ \end{tabular}
}

\bigskip\bigskip

%Branch zero
\flushleft
\LARGE\textbf{BZ} \large \hfill Branch if zero

\kern-3pt
\noindent\rule{16.5cm}{0.4pt}
\normalsize

{\large
	\makebox[3.5cm][l]{Opcode:} \texttt{10100} \par
	\smallbreak
	\makebox[3.5cm][l]{Syntax:} \texttt{BZ rd, <LABEL>} \par
	\smallbreak
	\makebox[3.5cm][l]{Purpose:} Branch to the label specified in the assembly program if  \par
	\makebox[3.5cm][l]{  } rd is equal to zero. \par
	\smallbreak
	\makebox[3.5cm][l]{Operation:} if (rd == 0): \par
	\makebox[4cm][l]{  } PC $\leftarrow$ LABEL \par
	\medbreak\medbreak
	\makebox[3.5cm][l]{Condition Codes:} \begin{tabular}{lll} N \quad & Z \quad & V \\ \hline x & - & x \\ \end{tabular}
}

\pagebreak

%Branch negative
\flushleft
\LARGE\textbf{BN} \large \hfill Branch if negative

\kern-3pt
\noindent\rule{16.5cm}{0.4pt}
\normalsize

{\large
	\makebox[3.5cm][l]{Opcode:} \texttt{10101} \par
	\smallbreak
	\makebox[3.5cm][l]{Syntax:} \texttt{BN rd, <LABEL>} \par
	\smallbreak
	\makebox[3.5cm][l]{Purpose:} Branch to the label specified in the assembly program if  \par
	\makebox[3.5cm][l]{  } rd is less than zero. \par
	\smallbreak
	\makebox[3.5cm][l]{Operation:} if (rd $<$ 0): \par
	\makebox[4cm][l]{  } PC $\leftarrow$ LABEL \par
	\medbreak\medbreak
	\makebox[3.5cm][l]{Condition Codes:} \begin{tabular}{lll} N \quad & Z \quad & V \\ \hline x & - & x \\ \end{tabular}
}

%Halt
\flushleft
\LARGE\textbf{HALT} \large \hfill Stop program

\kern-3pt
\noindent\rule{16.5cm}{0.4pt}
\normalsize

{\large
	\makebox[3.5cm][l]{Opcode:} \texttt{10110} \par
	\smallbreak
	\makebox[3.5cm][l]{Syntax:} \texttt{HALT} \par
	\smallbreak
	\makebox[3.5cm][l]{Purpose:} Stop the program. \par
}
\end{document}
